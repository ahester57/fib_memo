
% CS 4710 HW 1

\documentclass{article}
\usepackage{titling}
\usepackage{multirow}
\usepackage{siunitx}	% for scientific notation

\setlength{\droptitle}{-15em}

\begin{document}

\title{CS3130 - Project 2}
\author{Austin Hester}
\date{$\pi$ + 0.01, 2017}
\maketitle


% This makes it so sections aren't automatically numbered
\makeatletter
\def\@seccntformat#1{%
	 \expandafter\ifx\csname c@#1\endcsname\c@section\else
	  \csname the#1\endcsname\quad
  \fi}
\makeatother

%%%	Begin		%%%

%% Results Section %%

\section{Results}


\begin{center}

\begin{tabular}{ | l | c | r | }

	\multicolumn{3}{ c }{Memoization} \\
	\hline
	 n & $f_n$ & Time \\ \hline  

	64 & 10610209857723 & 0.000 \\ 
	67 & 44945570212853 & 0.000 \\ 
	71 & 308061521170129 & 0.000 \\ 
	73 & 806515533049393 & 0.000 \\ 
	79 & 14472334024676221 & 0.000 \\ 
	83 & 99194853094755497 & 0.000 \\ 
	89 & 1779979416004714189 & 0.000 \\ 
	93 & 12200160415121876738 & 0.000 \\ 
	94 & 1293530146158671551 & 0.000 \\ 
	100 & 3736710778780434371 & 0.000 \\ 
	5555 & 4281172103992949353 & 0.000 \\ 
	55555 & 15241805839757988469 & 0.015 \\ 
	555555 & 2326674059843183362 & 0.155 \\ 
	5555555 & 4485202671650115913 & 1.531 \\ 
	\hline

\end{tabular}
\quad
\begin{tabular}{ | l | c | r | }

	\multicolumn{3}{ c }{Pure Recursion} \\
	\hline
	 n & $f_n$ & Time \\ \hline  

	16 & 987 & 0.000 \\ 
	19 & 4181 & 0.000 \\ 
	23 & 28657 & 0.000 \\ 
	27 & 196418 & 0.000 \\ 
	32 & 2178309 & 0.015 \\ 
	36 & 14930352 & 0.125 \\ 
	38 & 39088169 & 0.343 \\ 
	39 & 63245986 & 0.562 \\ 
	40 & 102334155 & 0.906 \\ 
	41 & 165580141 & 1.500 \\ 
	44 & 701408733 & 6.218 \\ 
	45 & 1134903170 & 10.046 \\ 
	48 & 4807526976 & 42.625 \\ 
	\hline

\end{tabular}

\end{center}



%% Analysis Section %%

\section{Analysis}

%	Recursion	%

\subsection{Recursion}

Bubble sort took (literally) exponentially more time as the number of elements increased. \\

	Based on $T(n) = \theta (n^2)$, I got a "sort constant," $c \approx \num{1.566e9}$ from $\frac{n^2}{t}$ \\

Although this "sort constant" varies slightly from run to run, it can be used to successfully predict the time a certain
execution will take. For example, I was able to predict a bubble sort of 2 million integers would take 2547s using
the equation $t \approx \frac{n^2}{1.56e9}$, and the actual time ended up being 2571s.

\subsubsection{Average-case}

Average case  \\

	$T(n) = \theta (n^2)$

\newpage

%	Memoization		%

\subsection{Memoization}

Stops being right due to limits of unsigned long long at $94$.


	From $T(n) = \theta (n \lg{n})$,the "sort constant," $c \approx \num{4.270e7}$ from $\frac{n \lg{n}}{t}$ \\



\subsubsection{Average-case}

Average case  \\

	$T(n) = \theta (n^2)$



%% Conclusion %%

\section{Conclusion}

Don't use bubble sort unless you are using it to check to make sure an seemingly sorted list is completely sorted.

\end{document}














