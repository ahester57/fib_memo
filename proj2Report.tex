
% CS 4710 HW 1

\documentclass{article}
\usepackage{titling}
\usepackage{multirow}
\usepackage{siunitx}	% for scientific notation

\setlength{\droptitle}{-10em}

\begin{document}

\title{CS 3130 - Project 2}
\author{Austin Hester}
\date{$\pi$ + 0.01, 2017}
\maketitle


% This makes it so sections aren't automatically numbered
\makeatletter
\def\@seccntformat#1{%
	 \expandafter\ifx\csname c@#1\endcsname\c@section\else
	  \csname the#1\endcsname\quad
  \fi}
\makeatother

%%%	Begin		%%%

%% Results Section %%

\section{Results}


\begin{center}
\begin{tabular}{ | l | c | r | }
	\hline
	\multicolumn{3}{ |c| }{Pure Recursion} \\
	\hline
	$n$ & $f_n$ & Time (s) \\ \hline  

	16 & 987 & 0.000 \\ 
	19 & 4181 & 0.000 \\ 
	23 & 28657 & 0.000 \\ 
	32 & 2178309 & 0.015 \\ 
	36 & 14930352 & 0.125 \\ 
	38 & 39088169 & 0.343 \\ 
	40 & 102334155 & 0.906 \\ 
	41 & 165580141 & 1.500 \\ 
	44 & 701408733 & 6.218 \\ 
	45 & 1134903170 & 10.046 \\ 
	48 & 4807526976 & 42.625 \\ 
	55 & 139583862445 & 1245.358 \\ 
	64 & ???? & 5+ hrs \\
	\hline

\end{tabular}
\quad
\begin{tabular}{ | l | c | r | }
	\hline
	\multicolumn{3}{ |c| }{Memoization} \\
	\hline
	$n$ & $f_n$ & Time (s) \\ \hline  

	64 & 10610209857723 & 0.000 \\ 
	67 & 44945570212853 & 0.000 \\ 
	71 & 308061521170129 & 0.000 \\ 
	73 & 806515533049393 & 0.000 \\ 
	79 & 14472334024676221 & 0.000 \\ 
	83 & 99194853094755497 & 0.000 \\ 
	89 & 1779979416004714189 & 0.000 \\ 
	93 & 12200160415121876738 & 0.000 \\ 
	94 & 1293530146158671551 & 0.000 \\ 
	100 & 3736710778780434371 & 0.000 \\ 

	3600000 & 3525674565886733824 & 0.999 \\ 

	22222 & 834908324885882015 & 0.015 \\ 
	222222 & 6438811801179161448 & 0.062 \\ 
	2222222 & 11902358359429960313 & 0.609 \\ 
	22222222 & 14468215713834216991 & 6.092 \\ 
	222222222 & 6495968894011374952 & 61.609 \\ 

	5555 & 4281172103992949353 & 0.000 \\ 
	55555 & 15241805839757988469 & 0.015 \\ 
	555555 & 2326674059843183362 & 0.155 \\ 
	5555555 & 4485202671650115913 & 1.531 \\ 
	55555555 & 18256636889160166581 & 15.421 \\ 
	555555555 & 11757213551281829122 & 154.187 \\ 

	\hline

\end{tabular}
\end{center}

\newpage

%% Analysis Section %%

\section{Analysis}

%	Recursion	%

\subsection{Recursion}

\hspace{1em} Computing the Fibonacci numbers purely recursively has an order of growth in $\theta (1.618^{n+1})$. \\

	$T(n) = \Theta (1.618^{n+1}) \Rightarrow$ I got a "sort constant," $c \approx \num{4.1e8}$ from $\frac{1.618^{n+1}}{t}$ \\

Although this "sort constant" varies slightly from run to run, it can be used to successfully predict the time a certain
execution will take. For example, I was able to predict computing the $48^{th}$ Fibonacci number would take 42.38s using
the equation $t \approx \frac{1.618^{n+1}}{4.1e8}$, and the actual time ended up being 42.625s. \\

Using this method, I predict that computing the $64^{th}$ Fibonacci number will take $93500s \approx 26$ HOURS.



%	Memoization		%

\subsection{Memoization}

\hspace{1em} Stops being right due to limits of unsigned long long at $94$. \\

	From $T(n) = \Theta (n)$, I got a "sort constant," $c \approx \num{3.6e6}$ from $\frac{n}{t}$ \\

Using $c$, we can predict the time taken, in seconds, of the computation of the $n^{th}$ Fiboncci number with: 

\begin{center}
$t \approx {n} \div (\num{3.6e6})$
\end{center}

And indeed, computing the $3,600,000^{th}$ Fibonacci number took $0.999s$. \\

Interestingly, the Fibonacci numbers consisting of a certain number of $5s$ have running times:

\begin{center}
$t \approx 1.5 \times 10^{m}$, where $m = \# of digits - 7$
\end{center}

With $2s$:

\begin{center}
$t \approx 6 \times 10^{m}$, where $m = \# of digits - 8$
\end{center}

Which confirms a linear growth rate.
%% Conclusion %%

\section{Conclusion}

\hspace{1em} Don't waste energy trying to compute the $64^{th}$ Fiboncci number purely recursively.

\end{document}














